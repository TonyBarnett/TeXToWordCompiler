\documentclass[a4paper,fleqn]{article}
\usepackage{inputenc}
\usepackage{amsmath}
\usepackage{amsfonts}
\usepackage{amssymb}
\usepackage{setspace}
\usepackage{parskip}
\usepackage{xspace}
\usepackage{float}
\usepackage[explicit]{titlesec}
\usepackage[pdftex]{graphicx}
	\graphicspath{{./images/}}


% S T A N D A R D   C O M M A N D S
\renewcommand{\leq}{\leqslant}
\newcommand{\ie}{\emph{i.e.,\@\xspace}}
\newcommand{\etc}{\emph{etc.\@\xspace}} 
\newcommand{\cf}{\emph{cf.\@\xspace}}
\newcommand{\eg}{\emph{e.g.,\@\xspace}}
\newcommand{\etal}{\emph{et al.\@\xspace}}

% L O C A L   S T Y L E

% Margins
\usepackage{a4wide}

% Section title formatting
\let\stdsection\section
\renewcommand{\section}{\newpage\stdsection}

%referencing in the harvard style
\usepackage[round,authoryear,comma]{natbib}
\renewcommand{\cite}{\citep}
% T O P M A T T E R
\author{Anthony Barnett
\\
\\Student number: 16007896}
\title{Sustainable design and construction coursework}
\date{}
%\date{\today}

% Other
\usepackage{longtable}

\widowpenalty=3000 % prevent 'widows' and 'orphans'
\clubpenalty=3000

% D O C U M E N T
\begin{document}
\maketitle
%Breeam
%Code for sustainable homes
%Building regulations part L
\section*{Part 1: Describe the various environmental, sustainability and building energy standards/
	laws/ regulations/ codes of practice with which you are required to comply.}
%stdSection suppresses the linebreak before it... saves having just a section title on a page
\stdsection*{a: the existing city centre building, assuming a large building of over 1,000m$^2$}

Regulations and standards in the UK for the sustainable modification of existing buildings are:
\begin{itemize}
	\item building regulations part L1B \cite{Communities&LocalGovernment2010a}, and  
		L2B \cite{Communities&LocalGovernment2010c}, 
	\item parts six and seven of the building regulations \cite{Stunell2010} and, 
	\item ska \cite{Brown2013}
\end{itemize}

\emph{Building regulations part L1B} detail sustainable measures required when modifying 
an existing residential building. \emph{Building regulations part L2B} detail the measures 
required when modifying an existing building. The guidelines for building 
regulations part L were updated in 2012 \cite{Communities&LocalGovernment2013}. Part L of 
building regulations specify five criteria \cite{RichardsPartingtonArchitects2011}:
\begin{enumerate}
	\item achieving the Target Emission Rate (TER), 
	\item limits on design flexibility, 
	\item limiting the effect of solar gain in summer, 
	\item building performance consistent with Design Emissions Rate (DER) and, 
	\item provisions for energy efficient operation of the dwelling. 
\end{enumerate}

The \emph{European Commission (EC) 20-20-20} set targets of reducing greenhouse gas emissions by 20\% of 1990 
levels, and using 20\% renewable energy sources by 2020. To achieve this a target emissions rate was set for 
buildings. To calculate the TER of a non-domestic building the emissions of a \emph{notional building} of the 
same size and shape are calculated using modelling software, which complies with the National Calculation 
Methodology (NCM) \cite{Communities&LocalGovernment2008}.

The design flexibility constraint is enforced with the intention of forcing new build projects to be more 
environmentally friendly. An example of constrained flexibility is that all materials used must 
not exceed a specified U-value, dependant on the application of the material (party wall, exterior wall 
\etc)

To ensure the emissions of the building during the use-phase do no greatly exceed the DER, Post Occupancy 
Evaluations (POE) are required. The intention of a POE is to ensure the building caters for all the requirements 
of the building occupants. If a building does not fulfil the requirements of its occupants, it is likely that 
the emissions of the building will increase. For example, if the temperature control does not provide an adequate 
temperature, the occupants of the building will likely use space heaters which will increase the electricity 
consumption of the building.

\emph{Parts six and seven of the building regulations} detail requirements on energy and water efficiency. Mandatory 
restrictions are placed on water consumption. Methods for calculation of CO$_2$ emissions and energy efficiency are 
also specified against which a building is assessed.

\emph{Ska} is a retrofit-only rating system and is maintained by the Royal Institute of Chartered Surveyors (RICS). 
Ska provides a list of targets against which a building is evaluated. Not all parts of the list are applicable so 
\emph{Scoping measures} are used, \ie choosing which aspects of ska are applicable to the project. Bronze, silver, 
or gold is awarded for achieving 25\%, 50\%, and 75\% respectively.


\section*{b: the new-build development}

Sustainable regulations and standards in the UK for the new buildings:
\begin{itemize}
	\item Building Research Establishment's Environmental Assessment Method (BREEAM) \cite{BRE2012}
	\item Building regulations part L1A \cite{Communities&LocalGovernment2010}, and  
		L2A \cite{Communities&LocalGovernment2010b} and, 
	\item Code for Sustainable Homes (CSH) \cite{Communities&LocalGovernment2010d}
	\item Leadership in Energy and Environmental Design (LEED)
\end{itemize}

\emph{BREEAM} was developed by Building Research Establishment (BRE) as a means of assessing the environmental 
footprint of a building. The result of a BREEAM assessment is a building rating based on the percentage success. 
The ratings are given in table \ref{tab:BREEAMMarks}. To achieve these marks are awarded for different sectors 
given in table \ref{tab:BREEAMSectors}.

\emph{Building regulations part L1A and L2A} detail sustainability measures to be carried out on new-build 
properties. L1A details sustainability measures for residential buildings, whereas L2A details sustainability 
measures for non-residential buildings.

\emph{Code for sustainable homes} is a UK government standard which determines the environmental 
credentials of a new building based on a weighted set of criteria listed in table \ref{tab:cshWeighting}. The 
Building is scored against a number of categories, listed in table \ref{tab:cshCategories}. Additionally there 
are 3 non-weighted criteria that need to be met, otherwise the rating is not awarded. They are: Environmental 
impact of materials, management of surface water run-off from developments, and storage of non-recyclable waste 
and recyclable household waste.
If the building meets these criteria, it is awarded a grade from level 1 to 6. Current UK building regulations 
advise that all new buildings meet code for sustainable homes Level 3 at a minimum. 

Additional requirements of the different levels for CSH are given in table \ref{tab:CSHCO2Water}. This shows that 
it is possible to achieve only CSH level 3 without the use of any renewable technology. To achieve CSH level of 
4 or greater, one or more renewable technologies is required. This is because the TER is calculated on the ideal 
building using non-renewable technology, however to achieve a DER of 25\% or greater above the TER, on or more 
renewable technologies is required.

\emph{LEED} is an American standard that has been adopted in England to a limited extent. Similar to BREEAM, 
LEED Uses categories in which the performance of a building is scored however, unlike BREEAM, the score against 
certain categories is mandatory to achieve accreditation. The score is summed to give an overall score. A list of 
requirements for the scores is given in table \ref{tab:LEEDScoreRequirements} and the categories are given in 
table \ref{tab:LEEDCategories}.

\pagebreak
\section*{Part 2: Select \textbf{two} out of the environmental, sustainability and building 
	energy standards/ laws/ regulations/ codes of practice from the list of those you described 
	in your answer to question 1).}

%stdSection suppresses the linebreak before it... saves having just a section title on a page
\stdsection*{i: Describe the procedures, practices, building assessments, measurements,
	monitoring, recording, reporting and mitigation actions you will instruct to be taken
	in order to ensure that these buildings comply.}


\subsection*{a: the existing city centre building, assuming a large building of over 1,000m$^2$}
\subsubsection*{Ska}
%Procedures and practices

%Building assessment
To assess the performance of the measures that have been put in place an online tool is available \cite{ska2013}, 
however, to attain a ska rating a formal assessment is required by a ska assessor. 

The assessor will evaluate the project at 3 stages: \emph{design/ planning}, \emph{delivery/ construction}, and 
\emph{Occupancy stage}. In the design/ planning stage, the scope of the project is defined, \ie what is to be 
included in the project, an indicative rating is given based on proposed work. In the delivery/ construction 
stage evidence will need to be provided to prove to the ska assessor that the proposed work will be completed.
The occupancy stage is optional, the assessor inspects the building a year after completion of the work to 
compare the performance of the building against the expected performance.
%Measurements and monitoring

%Recording and reporting

%Mitigation actions

\subsubsection*{Building regulations part L}
%Procedures and practices
Part of the requirement of building regulations part L if the retrofit work is on a commercial building with 
useful floor area greater than 1000m$^2$, then 10\% of the funding for the project must be on 
\emph{consequential improvements}. Consequential improvements are systems put in place to improve the energy 
efficiency of the building. Examples of consequential improvements are replacing windows for double glazing, or 
upgrading the lighting systems \etc I will therefore instruct that, as part of the building work, 

%Building assessment
The building assessment is done as part of the planning application. The applicant assesses the performance of the 
building design during the planning phase of the project. Supporting documentation and evidence is uploaded to the 
government website, and are used to determine the suitability of the building. The building is assessed on BER, 
the limiting effects of solar gain, and energy efficiency. The assessment only applies to the domestic development 
and so will not be applied to the commercial development.

%Measurements and monitoring


%Recording and reporting
All improvements to the building need to be recorded in the building's log book. In the log book a record is 
required of all work done to the building, as well as the work carried out as part of the consequential 
improvements. The application for compliance is done as part of the application for planning permission and is 
undertaken on the government website.

%Mitigation actions



\subsection*{b: the new build development}
\subsubsection*{Code for Sustainable Homes (CSH)}
%Procedures and practices
The main procedures that I will instruct to be taken are to assess the suitability of renewable energy and 
incorporate them into the design of the building. For example implementing solar PV will reduce the Dwelling 
emissions rate. It can also possibly reduce the NOx emissions when combined with a ground source heat pump to 
provide zero carbon heating.

%Building assessment
The assessment is done by a third party. I would instruct that the building project be registered with a qualified 
assessor who would assess the building at the planning stage, and the post construction stage. This will ensure, 
firstly that the design of the building will qualify for the rating, and secondly that the design is correctly 
implemented.

%Measurements and monitoring


%Recording and reporting
The recording and reporting are largely done by the organisation carrying out the assessment. The assessor is 
responsible for registering the building project with the code service provider, and providing the assessment. 

%Mitigation actions


\subsubsection*{Building regulations part L}
%Procedures and practices


%Building assessment


%Measurements and monitoring
I would instruct that the energy consumption of the new building is monitored during occupancy. This is to 
ensure that energy saving measures introduced during the building phase, for example if solar PV were placed 
on the roof as an alternate source of building power, are effective and being fully utilised.

%Recording and reporting
Records and reporting of all measures taken towards compliance to part L of the building regulations are to be 
recorded in the buildings log book. It is advised that the logbook is regularly inspected to ensure that the 
information contained within it is up to date and accurate, and that procedures and practices contained within 
it are fully complied with.

%Mitigation actions
Prior to the start of the project it will be advised that soil from the site is tested for potential contagions. 
An environmental survey will also be carried out on the local ecosystem to ensure that the new building will 
not disrupt any of the local wildlife. Pending the results of the environmental survey mitigation action will 
potentially be advised to reduce the effect of the new building and further protect the local wildlife.


\section*{ii: Explain your justifications for the instructions that you gave in question part i.,
	paying particular attention to the adverse consequences that you seek to avoid to
	minimise your liabilities}


\subsection*{a: the existing city centre building, assuming a large building of over 1,000m$^2$}
\subsubsection*{Ska}


\subsubsection*{Building regulations part L}


\subsection*{b: the new build development}
\subsubsection*{CSH}


\subsubsection*{Building regulations part L}

\bibliographystyle{dcu}
\bibliography{../../../../../SDCO,Additional}

\appendix

\section{Appendix}

\subsection{Details about CSH}
\begin{table}[H]
	\begin{center}
		\begin{tabular}{| l | r | r | r |}
		\hline
			\multicolumn{1}{|l|}{\textbf{Categories of}} & \multicolumn{1}{l|}{\textbf{Total Credits in}} & 
				\multicolumn{1}{l|}{\textbf{Weighting Factor}} & \multicolumn{1}{l|}{\textbf{Approximate Weighted}}\\
			\multicolumn{1}{|l|}{\textbf{Environmental Impact}} & \multicolumn{1}{l|}{\textbf{each Category}} & 
				\multicolumn{1}{l|}{\textbf{(\% points contribution)}} & \multicolumn{1}{l|}{\textbf{Value of each Credit}}\\
			\hline
			Energy and CO2 Emissions & 31 & 36.4\% & 1.17\\
			\hline
			Water & 6 & 9.0\% & 1.50\\
			\hline
			Materials & 24 & 7.2\% & 0.30\\
			\hline
			Surface Water Run-off & 4 & 2.2\% & 0.55\\
			\hline
			Waste & 8 & 6.4\% & 0.80\\
			\hline
			Pollution & 4 & 2.8\% & 0.70\\
			\hline
			Health and Well-being & 12 & 14.0\% & 1.17\\
			\hline
			Management & 9 & 10.0\% & 1.11\\
			\hline
			Ecology & 9 & 12.0\% & 1.33\\
			\hline
		\end{tabular}
	\caption{Categories and weightings for code for sustainable homes. \cite{Communities&LocalGovernment2010d}}
	\label{tab:cshWeighting}
	\end{center}
\end{table}

\begin{longtable}[H]{| l | r | r |}
	\hline
	\textbf{Code Categories} & \textbf{Available} & \textbf{Category}\\
	& \textbf{Credits} & \textbf{Weighting Factor}\\
	
	\hline
	\endhead % before this appears at the start of each table part (\ie when a new page is started)
	
	\hline
	\endfoot % before this appears at the end of each table part (\ie when a page is finished)
	
	\endlastfoot % before this appears at the end of the table
	\textbf{Energy and CO2 Emissions} & \textbf{31} & \textbf{36.40}\\*
	Dwelling emission rate & 10 &\\*
	Fabric energy efficiency & 9 &\\*
	Energy display devices & 2 &\\*
	Drying space & 1 &\\*
	Energy labelled white goods & 2 &\\*
	External lighting & 2 &\\*
	Low and zero carbon technologies & 2 &\\*
	Cycle storage & 2 &\\*
	Home office & 1 &\\*
	& &\\
	\textbf{Water} & \textbf{6} & \textbf{9.00}\\*
	Indoor water use & 5 &\\*
	External water use & 1 &\\*
	& &\\
	\textbf{Materials} & \textbf{24} & \textbf{7.20}\\*
	Environmental impact of materials & 15 &\\*
	Responsible sourcing of materials – basic building elements & 6 &\\*
	Responsible sourcing of materials – finishing elements & 3 &\\*
	& & \\*
	\textbf{Surface Water Run-off} & \textbf{4} & \textbf{2.20}\\*
	Management of surface water run-off from developments & 2 &\\*
	Flood risk & 2 &\\*
	& & \\
	\textbf{Waste} & \textbf{8} & \textbf{6.40} \\*
	Storage of non-recyclable waste and recyclable household waste & 4 & \\*
	Construction site waste management & 3 & \\*
	Composting & 1 &\\*
	& &\\
	\textbf{Pollution} & \textbf{4} & \textbf{2.80}\\*
	Global warming potential (GWP) of insulants & 1 &\\*
	NOx emissions & 3 &\\*
	& &\\
	\textbf{Health \& Well-being} & \textbf{12} & \textbf{14.00}\\*
	Daylighting & 3 &\\*
	Sound insulation & 4 &\\*
	Private space & 1 &\\*
	Lifetime Homes & 4 &\\*
	& &\\
	\textbf{Management} & \textbf{9} & \textbf{10.00}\\*
	Home user guide & 3 &\\*
	Considerate Constructors Scheme & 2 &\\*
	Construction site impacts & 2 &\\*
	Security & 2 &\\*
	& &\\
	\textbf{Ecology} & \textbf{9} & \textbf{12.00}\\*
	Ecological value of site & 1 &\\*
	Ecological enhancement & 1 &\\*
	Protection of ecological features & 1 &\\*
	Change in ecological value of site & 4 &\\*
	Building footprint & 2 &\\*
	& &\\
	\hline
	\caption{Scores for code for sustainable homes, broken down by category. 
		\cite{Communities&LocalGovernment2010d}}
	\label{tab:cshCategories}
\end{longtable}
		
\begin{table}[H]

\begin{table}[H]
	\begin{center}
	\begin{tabular}{| l | c | c |}
	\hline
	\textbf{Code level} & \textbf{Minimum percentage improvement } & \textbf{Maximum indoor water }\\
	 & \textbf{in DER over TER} & \textbf{consumption (litres/day)} \\
	 \hline
	 Level 1 & 0\% & 120 \\
	 Level 2 & 0\% & 120 \\
	 Level 3 & 0\% & 105 \\
	 Level 4 & 25\% & 105 \\
	 Level 5 & 100\% & 80 \\
	 Level 6 & net zero CO$_2$ emissions & 80 \\
	 \hline
	\end{tabular}
	\caption{Net carbon dioxide emissions and water consumption for each of the CSH levels. Adapted 
		from \cite{Communities&LocalGovernment2010d}}
	\label{tab:CSHCO2Water}
	\end{center}
\end{table}
	\begin{center}
	\begin{tabular}{|c|l|}
		\hline
		\textbf{Total Percentage Points Score} & \textbf{Code Levels}\\
		\textbf{(equal to or greater than)} & \\
		\hline
		36 Points & Level 1\\
		\hline
		48 Points & Level 2\\
		\hline
		57 Points & Level 3\\
		\hline
		68 Points & Level 4\\
		\hline
		84 Points & Level 5\\
		\hline
		90 Points & Level 6\\
		\hline
	\end{tabular}
	\end{center}
	\caption{score requirements for code for sustainable homes}
	\label{tab:cshScoreRequirements}
\end{table}


\subsection{Details about BREEAM}
\begin{table}[H]
	\begin{center}
		\begin{tabular}{| l | r |}
			\hline
			\textbf{BREEAM rating} & \textbf{Minimum \% score}\\
			\hline
			Outstanding & 85\\
			Excellent & 70\\
			Very good & 55\\
			Good & 45\\
			Pass & 30\\
			\hline
		\end{tabular}
	\end{center}
	\caption{BREEAM ratings and their requirements adapted from \cite{BRE2012}}
	\label{tab:BREEAMMarks}
\end{table}

\begin{table}[H]
	\begin{center}
		\begin{tabular}{| l | c | c |}
			\hline
			\textbf{BREEAM sector} & \textbf{number of available credits} & \textbf{percentage of total}\\
			\hline
			Management & 22 & 17\% \\
			Health \& wellbeing & 10 & 8\% \\
			Energy & 30 & 23\% \\
			Transport & 9 & 7\% \\
			Water & 9 & 7\% \\
			Materials & 12 & 9\% \\
			Waste & 7 & 5\% \\
			Land use \& Ecology & 10 & 8\% \\
			Pollution & 13 & 10\% \\
			Innovation & 10 & 8\% \\
			\hline
			\textbf{Total} & \textbf{132} & \textbf{100\%} \\
			\hline
		\end{tabular}
	\end{center}
	\caption{Number of available credits for each BREEAM sector adapted from \cite{BRE2012}}
	\label{tab:BREEAMSectors}
\end{table}


\subsection{Details about LEED}
\begin{table}[H]
	\begin{center}
	\begin{tabular}{| l | c |}
		\hline
		\textbf{Award} & \textbf{minimum score}\\
		\hline
		Platinum & 80 \\
		Gold & 60 \\
		Silver & 50 \\
		certified & 40 \\
		\hline
	\end{tabular}
	\end{center}
	\caption{Requirements for scoring levels in LEED, adapted from USGBC website \cite{LEED2013}}
	\label{tab:LEEDScoreRequirements}
\end{table}

\begin{longtable}[H]{| l | c |}
	\hline
	\textbf{Code Categories} & \textbf{Available Credits}\\
	
	\hline
	\endhead % before this appears at the start of each table part (\ie when a new page is started)
	
	\hline
	\endfoot % before this appears at the end of each table part (\ie when a page is finished)
	
	\endlastfoot % before this appears at the end of the table
	\textbf{Sustainable sites} & \textbf{26} \\* % the * blocks the pagebreak
	Construction activity pollution prevention & REQUIRED \\*
	Site selection & 1 \\*
	Development density and community connectivity & 5 \\*
	Brownfield redevelopment & 1 \\*
	Alternative transportation - public transportation access & 6 \\*
	Alternative transportation - bicycle storage and changing rooms & 1 \\*
	Alternative transportation - low-emitting and fuel-efficient vehicles & 3 \\*
	Alternative transportation - parking capacity & 2 \\*
	Site development - protect or restore habitat & 1 \\*
	Site development - maximize open space & 1 \\*
	Stormwater design - quantity control & 1 \\*
	Stormwater design - quality control & 1 \\*
	Heat island effect - nonroof & 1 \\*
	Heat island effect - roof & 1 \\*
	Light pollution reduction & 1 \\*
	& \\
	\textbf{water efficiency}  & \textbf{10} \\*
	Water use reduction & REQUIRED \\*
	Water efficient landscaping & 4 \\*
	Innovative wastewater technologies & 2 \\*
	Water use reduction & 4 \\*
	& \\
	\textbf{Energy \& atmosphere} & \textbf{35} \\*
	Fundamental commissioning of building energy systems & REQUIRED \\*
	Minimum energy performance & REQUIRED \\*
	Fundamental refrigerant Mgmt & REQUIRED \\*
	Optimize energy performance & 19 \\*
	On-site renewable energy & 7 \\*
	Enhanced commissioning & 2 \\*
	Enhanced refrigerant Mgmt & 2 \\*
	Measurement and verification & 3 \\*
	Green power & 2 \\*
	& \\
	\textbf{Material \& resources} & \textbf{14} \\*
	Storage and collection of recyclables & REQUIRED \\*
	Building reuse - maintain existing walls, floors and roof & 3 \\*
	Building reuse - maintain interior nonstructural elements & 1 \\*
	Construction waste Mgmt & 2 \\*
	Materials reuse & 2 \\*
	Recycled content & 2 \\*
	Regional materials & 2 \\*
	Rapidly renewable materials & 1 \\*
	Certified wood & 1 \\*
	& \\
	\textbf{Indoor environment quality} & \textbf{15} \\*
	Minimum IAQ performance & REQUIRED \\*
	Environmental Tobacco Smoke (ETS) control & REQUIRED \\*
	Outdoor air delivery monitoring & 1 \\*
	Increased ventilation & 1 \\*
	Construction IAQ Mgmt plan - during construction & 1 \\*
	Construction IAQ Mgmt plan - before occupancy & 1 \\*
	Low-emitting materials - adhesives and sealants & 1 \\*
	Low-emitting materials - paints and coatings & 1 \\*
	Low-emitting materials - flooring systems & 1 \\*
	Low-emitting materials - composite wood and agrifiber products & 1 \\*
	Indoor chemical and pollutant source control & 1 \\*
	Controllability of systems - lighting & 1 \\*
	Controllability of systems - thermal comfort & 1 \\*
	Thermal comfort - design & 1 \\*
	Thermal comfort - verification & 1 \\*
	Daylight and views - daylight & 1 \\*
	Daylight and views - views & 1 \\*
	& \\
	\textbf{introduction/ other} & \textbf{6} \\*
	Innovation in design & 5 \\*
	LEED Accredited Professional & 1 \\*
	Introduction/OtherIntroduction/Other & \\*
	& \\*
	\textbf{Regional priority} & \textbf{4} \\*
	Regional priority & 4 \\*
	& \\*
	\hline
	\textbf{TOTAL} & \textbf{110} \\*
	\hline
	\caption{Points available for each LEED category, adapted from USGBC website \cite{LEED2013}}
	\label{tab:LEEDCategories}
\end{longtable}

\end{document}
